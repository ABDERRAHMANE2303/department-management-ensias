\section{Introduction}
Dans ce chapitre, nous présentons le contexte général de notre projet. Nous commencerons par une présentation de la problématique du projet et de ses objectifs. Ensuite, nous élaborerons la partie du déroulement du projet.
\newpage
\section{Présentation du Projet}
Ce projet vise à développer une plateforme centrale pour bien présenter et gérer les départements de l'ENSIAS.

\section{Problématique}
Le site web officiel de l'ENSIAS ne suffit pas pour traduire la qualité de l'école dans le domaine de l'informatique, en particulier en ce qui concerne la visibilité de ce qu'offrent les départements de l'école.

En particulier, il y a un manque de centralisation des données. Les formations des filières sont présentées dans une section du site web actuel qui n'a aucune liaison avec la section des départements, qui ne contient pour le moment que des informations sur les professeurs appartenant à un département spécifique.

D'autre part, la présentation elle-même doit être améliorée pour être plus attrayante. Le design de la plateforme actuelle manque de modernité et des principes simples de l'UX/UI.

\section{Objectifs du projet}
L'objectif principal de notre plateforme est de centraliser les informations concernant les départements de l'ENSIAS, ainsi que de les présenter d'une manière plus moderne et attrayante, attirant ainsi les étudiants, les enseignants et les partenaires possibles. Pour ce faire, nous avons défini les objectifs suivants :
\begin{enumerate}
    \item \textbf{Centraliser la présentation des informations concernant les départements de l'ENSIAS :} Ces informations concernent :
    \begin{itemize}
        \item Les professeurs appartenant à chaque département.
        \item Les chefs des départements de chaque filière.
        \item Les filières disponibles ainsi que les descriptions détaillées de chacune.
        \item Les partenariats académiques (c'est-à-dire les offres de double diplomation, échanges...).
        \item Les ressources appartenant à chaque département (à être détaillées).
    \end{itemize}
    \item \textbf{Être un moyen de gestion des départements :} Cette gestion contiendra des parties comme l'ajout de nouveaux professeurs par les chefs des filières...
    \item Afficher des statistiques afin d'avoir une visualisation plus directe des départements.
\end{enumerate}

